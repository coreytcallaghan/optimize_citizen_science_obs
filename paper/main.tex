\documentclass[9pt,twocolumn,twoside,lineno]{pnas-new}
% Use the lineno option to display guide line numbers if required.

\templatetype{pnasresearcharticle} % Choose template
% {pnasresearcharticle} = Template for a two-column research article
% {pnasmathematics} %= Template for a one-column mathematics article
% {pnasinvited} %= Template for a PNAS invited submission

\title{Optimizing the future of biodiversity sampling by citizen scientists}

% Use letters for affiliations, numbers to show equal authorship (if applicable) and to indicate the corresponding author
\author[a,b,1]{Corey T. Callaghan}
\author[c,]{Alistair G. B. Poore}
\author[b,a]{Richard E. Major}
\author[b,a]{Jodi J. L. Rowley}
\author[a,c]{William K. Cornwell}

\affil[a]{Centre for Ecosystem Science, School of Biological, Earth and Environmental Sciences, UNSW Sydney, Sydney, 2052, NSW, Australia}
\affil[b]{Australian Museum Research Institute, Australian Museum, Sydney, 2000, NSW, Australia}
\affil[c]{Ecology and Evolution Research Centre, School of Biological, Earth and Environmental Sciences, UNSW Sydney, Sydney, 2052, NSW, Australia}

% Please give the surname of the lead author for the running footer
\leadauthor{Callaghan}

% Please add here a significance statement to explain the relevance of your work
\significancestatement{Citizen scientists are increasingly relied on for monitoring of biodiversity, but we need to better understand how to optimally sample biodiversity in time and space for specific outcomes, relying on citizen science data. We provide a framework which optimizes citizen science sampling, specific to an intended statistical outcome.}

% Please include corresponding author, author contribution and author declaration information
\authorcontributions{Author contributions: C.T.C., W.K.C. designed research, performed research, and analyzed research. C.T.C., W.K.C., J.J.L.R., A.G.B.P., and R.E.M. wrote the paper.}
\authordeclaration{The authors declare no conflict of interest.}

\correspondingauthor{\textsuperscript{1}To whom correspondence should be addressed. E-mail: c.callaghan@unsw.edu.au}

% Keywords are not mandatory, but authors are strongly encouraged to provide them. If provided, please include two to five keywords, separated by the pipe symbol, e.g:
\keywords{biodiversity $|$ conservation $|$ citizen science $|$ optimal sampling $|$ eBird}

\begin{abstract}
Monitoring biodiversity trends in space and time is critical for conservation assessment and threat mitigation. Increasingly, scientists and conservationists are turning to citizen science data to extract information about population trends. But, citizen science data are generally 'noisy' with a suite of biases, seen these biases are seen as a necessary hurdle when using these data. In order to continue to rely on citizen science in the future of biodiversity monitoring, we aimed to provide a generalized framework and tested how a suite of temporal and spatial sampling biases influence the marginal value of a given citizen science observation --- borrowing from the theory of statistical leverage. We find that for a given site, the days since the last sample and the distance from the nearest sampled grid are positively associated with the marginal value of a citizen science observation. We introduce a dynamic framework, whereby we can predict expected marginal value (i.e., expected leverage) for a given site in space and time. Ultimately, we envision an approach where citizen scientists are guided and incentivized to submit sampling at sites with the highest expected marginal value.
\end{abstract}

\dates{This manuscript was compiled on \today}
\doi{\url{www.pnas.org/cgi/doi/10.1073/pnas.XXXXXXXXXX}}

\begin{document}

\maketitle
\thispagestyle{firststyle}
\ifthenelse{\boolean{shortarticle}}{\ifthenelse{\boolean{singlecolumn}}{\abscontentformatted}{\abscontent}}{}

% If your first paragraph (i.e. with the \dropcap) contains a list environment (quote, quotation, theorem, definition, enumerate, itemize...), the line after the list may have some extra indentation. If this is the case, add \parshape=0 to the end of the list environment.
\dropcap{A}ssessing biodiversity trends in space and time is essential for conservation \cite{harrison2014assessing, wilson2011modelling, mcmahon2011improving, honrado2016fostering, yoccoz2001monitoring} \parshape=0. Reliable biodiversity trend estimates, at multiple spatial scales \cite{soberon2007assessing}, allow us to track our global progress in curbing biodiversity loss \cite{harrison2014assessing}. Unsurprisingly, reliable trend estimates are best derived from well-designed surveys in space and time \cite{harrison2014assessing, vellend2017estimates, kery2009trend}, that are long-term \cite{lindenmayer2012value, magurran2010long}. But scientific funding for long-term ecological and conservation research is rapidly disappearing [REFS???]. Increasingly, government agencies, scientific researchers, and conservationists are turning to citizen science data to help inform the state of biodiversity at local \cite{callaghan2015efficacy, theobald2015global, sullivan2017using, loss2015linking}, regional \cite{barlow2015citizen, fox2011new}, and global scales \cite{chandler2017contribution, pocock2018vision, cooper2014invisible}.

Citizen science --- the cooperation between a range of experts and non-experts  --- is an incredibly diverse field \cite{jordan2015citizen}. Projects generally fall along a continuum based on the level of associated structure \cite{kelling2019using, welvaert2016citizen}, ranging from unstructured (e.g., opportunistic or incidental projects with little to no training required; iNaturalist) to structured (e.g., projects with specific objectives, rigorous protocols, and survey design; U.K. Butterfly Monitoring Scheme). The level of structure, in turn, influences the biases and data quality concerns of a particular project. For instance, observer skill \cite{kelling2015can}, time-of-day, number of participants in a group, and technological capabilities of a participant may influence the data collected by some, but not necessarily all, citizen science projects. Generalizeable among citizen science projects, however, are various spatial and temporal biases \cite{boakes2010distorted, bird2014statistical}. Observers submitting observations on weekends \cite{courter2013weekend}, sampling near roads and human settlements \cite{kelling2015taking}, and observers visiting known 'hotspots' for biodiversity \cite{geldmann2016determines} are all examples of spatial and temporal biases. These biases are not neccesarily restricted to citizen science projects, as our historical understanding of biodiversity in space and time is also biased, evidenced by natural history collections \cite{pyke2010biological, boakes2010distorted}. Indeed, many sampling methods have been proposed to optimally sample biodiversity \cite{etienne2005new, moreno2000assessing, colwell1994estimating, longino1997biodiversity, ferrarini2012biodiversity}, frequently dependent on spatial scale \cite{chase2013scale}. But little attention has been given to optimal sampling by citizen scientists \cite{harrison2014assessing}.

This is probably, in part, because these spatial and temporal biases have been seen as a 'necessary evil', associated with citizen science data \cite{parrish2018exposing}. Also, in the case of broad-scale biodiversity data collected at large voluminous scales (e.g., eBird), these various biases can generally be accounted for statistically \cite{isaac2014statistics, robinson2018correcting}, for instance, by filtering or subsetting data \cite{wiggins2011conservation}, pooling multiple data sources \cite{fithian2015bias}, or machine learning and hierarchical clustering techniques \cite{hochachka2012data, kelling2015taking}. Indeed, despite known biases, citizen science data have increased our knowledge of species distribution models \cite{bradsworth2017species, van2013opportunistic}, niche breadth \cite{tiago2017using}, biodiversity measurements \cite{stuart2017assessing, pocock2018vision}, phenological research \cite{la2014role, supp2015citizen}, invasive species detection \cite{pocock2017citizen, grason2018citizen}, and phylogeographical research \cite{bahls2014new, drury2019continent}. Still, estimating trends with citizen science data is best done with data from structured projects (i.e., less biases to account for) \cite{fox2011new}. But unstructured and semistructured projects are increasingly harnessed for trend detection \cite{walker2017using, kery2009trend, kery2010site, horns2018using, van2013occupancy, pagel2014quantifying}. From a conservation perspective, the goal is relatively straightforward: provide robust measures of a species' trend through time, a critical component of the IUCN Red List Index \cite{baillie2008toward}. The robustness of these trend estimates is critical, and the goal should be to continuously decrease the standard error of the trend estimates (e.g., Fig. 2). This is possible with citizen science data \cite{kery2010site, horns2018using, van2013occupancy, pagel2014quantifying} and is generally proportional to the number of observations, and appropriate sampling, through time (e.g., Video S1).

The number of citizen science projects which are focused on ecological and environmental monitoring is increasing \cite{pocock2017diversity, theobald2015global}, highlighting the potential that citizen science holds for the future of ecology, conservation, and natural resource management \cite{pocock2018vision, silvertown2009new, soroye2018opportunistic, mckinley2017citizen}. But a major obstacle in the future use of citizen science data remains understanding how to best extract information from 'noisy' citizen science datasets \cite{parrish2018exposing}.

As mentioned, this noise from citizen science \cite{bird2014statistical} can sometimes be alleviated using 'big-data' statistical approaches \cite{kelling2015taking}, but this is most applicable for data originating from large, successful citizen science projects --- with lots of data. What about projects that are just starting? Or projects focusing on taxa that are less popular with the general public \cite{mair2016explaining, ward2014understanding}? Are there optimal strategies for sampling in space and time for estimating biodiversity trends?

Here, we investigate these questions with a specific objective: investigate how spatial and temporal sampling influences trend detection of biodiversity. Our approach is site-specific and dynamic: we are interested in the parameters that influence the marginal value of a given citizen science observation for a particular day. We test our approach using a popular citizen science project --- eBird \cite{sullivan2009ebird} --- and > 5 million bird biodiversity observations from the Greater Sydney Region, NSW, Australia. First, we summarize the parameters in space and time which we hyptohesized would influence the marginal value of a citizen science observation.

\begin{itemize}
  \item \textbf{Whether a site was sampled}: if a site had been previously sampled or not. We predicted that unsampled sites would be marginally more valuable than sampled previously sampled sites.
  \item \textbf{Median sampling interval}: the median of the distribution of waiting times between samples at a site. We predicted that the median sampling interval would be positively associated with the value of a citizen science observation. I.e., observations from sites with high median waiting times would be more valuable than observations from sites with low median waiting times.
  \item \textbf{Days since last sample}: the number of days between samples at a site. We predicted that the days since the last sample would positively associate with the value of a citizen science observation.
  \item \textbf{Distance to the nearest sampled site}: the distance between the site in question and the nearest sampled site. We hypothesized that for the distance to the nearest sampled site would be positively associated with the value of a citizen science observation.
  \item \textbf{Nearest neighbor sampling interval}: the median sampling interval of the nearest neighbor. We hypothesized that this would be positively influence the value of a observation, whereby well-sampled areas (i.e., multiple sites near each other with low median sampling intervals) would have lower value observations.
  \item \textbf{Number of unique days sampled}: the total number of unique days sampled for a given site. We predicted that the total number of unique days would be positively associated with the value of an observation, whereby sites with lots of observations would receive value given the long-term data originating from them.
\end{itemize}

\section*{Results and Discussion}
We found weak evidence that visiting an unsampled site was marginally more valuable than visiting an already sampled grid cell (Fig. S1). To assess the robustness of these results, we investigated our predictions at four different grid sizes: 5, 10, 25, and 50 km$^{2}$ grids. For the 5 km (p=0.669), and 10 km (p=0.093) grid sizes, there was an insigificant relationship, but for the 25 km (p=0.035) grid size, there was a statistically significant relationship. At the 50 km grid sizes, the test was not possible because all grids had been sampled. This suggests that stratified sampling --- an approach which aims for equal sampling among grids or sites --- \cite{baillie2008toward, longino1997biodiversity} is not the most applicable approach for detecting trends using citizen science data. In other words, citizen scientists contributing to eBird, likely a large number who are qualified naturalists in their own right \cite{callaghan2018unnatural}, are already sufficiently sampling the biodiversity in space.

However, for those sampled sites, we found significant relationships between a number of our predicted parameters and the marginal value of a sampling event (Fig. 2). For all four grid sizes, days since the last sample in a grid was positively associated with the marginal value of a sampling event, and this parameter had the strongest effect size for all but the 5 km grid size models. Similarly, for all but the 10 km grid size, distance to the nearest sampled grid was positively associated with the marginal value of a sampling event. Surprisingly, the results for median sampling interval were negatively associated with the marginal value of a sampling event. Nearest neighbor sampling interval and the total number of uniquely sampled days had relatively weak influence on the value of a sampling event, compared with other parameters (i.e., smaller effect sizes). Together, these results suggest that there are possible improvements in biodiversity sampling for trend detection. Through time, the longer a grid goes unsampled (i.e., days since last sample), that grid's marginal value will continue to increase. Similarly, this is synergistic with the distance to the nearest sampled grid, whereby the further away from a sampled grid an observation is, the more valuable that observation is. The effect of distance was strongest at the smallest grid size, suggesting that when more grids (i.e., sites) are included, that the distance to the nearest sample is increasingly important for monitoring biodiversity trends.

The fact that median samplin interval of a site, as well as the number of unique days sampled at a site were generally insignificant suggests that the 'history' of a given site (i.e., the number and frequency of observations at that site) does not influence the dynamic marginal value of a sampling event.

We found generally consistent results, based on the grid size chosen, highlighting the influence of spatial scale of our approach. It is critical to track biodiversity trends at multiple spatial scales \cite{soberon2007assessing}, as biodiversity estimates sometimes change dependent on the spatial scale \cite{chase2013scale}. Our approach of relying on statistical leverage as a measure of the value of a citizen science observation can be used regardless of whether a citizen science project is global, regional, or carried out in the constraints of a local park. The site can be gridded --- the size of a grid, however, would be proportional to the spatial scale of the study --- and the framework implemented, updated dynamically.

Providing dynamic feedback to citizen science participants has proved successful for many citizen science projects \cite{rowley2019frogid, wiggins2011conservation, xue2016avicaching}. Generally, this feedback is in the form of leaderboards, presenting the number of submissions or number of unique species someone has contributed \cite{wood2011ebird}. We used our fitted models to predict the expected value of a given citizen science observation, dynamically, for any given day (e.g., Fig. 3, Video S2). Unsurprisingly, the grids with the highest expected leverage correlated negatively with human population density (Fig. S3). This approach required us to look backward first, using a model with all observations for 2018, based on dfBetas calculated from 2010-2018, in order to look forward and pedict the expected leverage for any given day. We envision a dynamic approach (Video S2) in the future of citizen science projects, which would ultimately guide participants to sites which \textit{should} be sampled on any given day --- or in a given week, month, or year. In this instance, leaderboards would move past numbers of species or submission and complementary leaderboards could be derived based on a participant's cumulative value to the citizen science dataset. Instead of participants preferrentially chasing specific species, this approach would guide participants to the sites with the highest expected marginal value. For example, we imagine visitor centers across the world at national parks or urban greenspaces providing their visitors on any given day a localized map showing which trail someone should visit if they are interested in contributing the greatest value to citizen science. The global pull of ecotourism \cite{sharpley2006ecotourism} is increasing exponentially, creating the potential for people to contribute to local biodiversity knowledge in areas that are traditionally undersampled, and with this framework, their observations can be maximized.

Here, our framework focused on a specific statistical outcome: trend detection. Many other ecological outcomes arise from citizen science, including species distribution models \cite{bradsworth2017species, van2013opportunistic}, phylogeographical research \cite{bahls2014new, drury2019continent}, invasive species detection \cite{pocock2017citizen, grason2018citizen}, or phenological research \cite{la2014role, supp2015citizen}. These potential outcomes will have different optimal sampling strategies in space and time, but can still me quantified in the same framework we introduce here. For example, an intended outcome of a spcecies distribution model would likely place greater value on observations from unsampled grids \cite{crawley2001scale} than for species trend detection. Our framework is robust, with the necessary piece of information being a statistical model derived based on an intended ecological/management outcome.

\subsection*{Conclusions}
Since eBird inception in 2002, citizen science contributors have collectively contributed > 30 million effort hours. And this is only one citizen science project, demonstrating the cumulative effort put-forth by citizen scientists. Arguably, citizen science will continue to shape the future of ecology and conservation, as it has substantially for the past couple centuries \cite{silvertown2009new}, with an increasingly critical role \cite{mckinley2017citizen, pocock2018vision} in the monitoring of biodiversity. But we need to look towards the future. Are there mechanisms we can put in place now which will increase our collective knowledge gleaned from citizen science datasets for biodiversity in the future? We found that there are general rules which could help to guide citizen science participants to better sampling in space and time: the distance to the nearest sampled site and the days since the last sample at a site both positively correlate with the marginal value of a citizen science observation. Moreover, we demonstrate a framework which practicioners can implement to better optimize their sampling designs, specific to a citizen science project and an intended goal.

\matmethods{ We tested our predictions throughout the Greater Sydney Region, delineating grids across the region of varying size: 5, 10, 25, and 50 km$^{2}$ (Fig. SX), where a grid represented a 'site'. We used the R statistical environment \cite{rcoreteam2018r} to carry out all analyses, relying heavily on the tidyverse \cite{wickham2017tidyverse}, ggplot2 \cite{wickham2016ggplot}, and sf \cite{pebesma2018sf} packages.

In order to test our predictions, we relied on the eBird basic dataset (version $ebd-relDec-2018$; available at: https://ebird.org/data/download), subsetting the data between January 1$^{st}$, 2010 to December 31$^{st}$, 2018. eBird is a particular successful citizen science project with > 600 million observations contributed by > 400 thousand participants, globally \cite{sullivan2009ebird, sullivan2014ebird, sullivan2017using}. eBird relies on volunteer birdwatchers who collect data in the form of 'checklists' --- a list of all species identified (audibly or visually) for given spatiotemporal coordinates. eBird relies on an extensive network of regional reviewers who are local experts of the avifauna \cite{gilfedder2019brokering} to ensure data quality \cite{sullivan2009ebird}.

\subsection*{Trend detection model} We first filtered the eBird basic dataset \cite{callaghan2017assessing, johnston2018estimates, la2014role}, by the following criteria: (1) we only included complete checklists, (2) only included terrestrial bird species, (3) removed any nocturnal checklists, (4) only included checklists which were > 5 minutes and < 240 minutes in duration, (5) only included checklists which travelled < 5 km or covered < 500 Ha, and (6) only included checklists which had > 4 species on it, as checklists with less than 4 species were likely to be targeted searches for particular species \cite{walker2017using, szabo2010regional}.

For any species with > 50 observations (N=235), we fit a generalized linear model using the 'glm' function in R, based on presence/absence \cite{walker2017using, horns2018using}. The models consisted of a continuous term for day, beginning on January 1$^{st}$, 2010, and a categorical term for county, providing a spatial component to the models (e.g., Fig. 1). We also included an offset term for the number of species seen on a given eBird checklist, accounting for temporal and spatial effort of that checklist \cite{szabo2010regional}. Each of the models were fit with a binomial family distribution. A total of 25,995 observations (i.e., eBird checklists) were used to fit each of the models.

\subsection*{Statistical leverage} Statistical leverage measures the influence of a particular observation on the independent variable \cite{cook1977detection}. In other words, it is a measure of how much a given observation influences the results of a statistical outcome. In our instance, we had multiple predictor variables in our GLMs, and so we used dfBeta as a measure of statistical leverage for each observation. dfBeta measures the change to the observed parameters of a new regression equation, after omitting the \textit{i$^{th}$} observation from the dataset \cite{belsley1980regression}. It follows the formula:

\begin{equation}
D F B E T A=\hat{\beta}-\beta_{(i)}=\frac{\left(X^{\prime} X\right)^{-1} X_{i} r_{i}}{1-h_{i}}
\end{equation}

where X is the predictor variable matrix, \textit{r} the residual vector, \textit{i h} the \textit{i$^{th}$} diagonal member, and \textit{i x} the \textit{i$^{th}$} line of matrix X. The value of dfBeta proportionally decreases with increasing number of observations.

In our case, each observation for a given species recieved a dfBeta value (i.e., each species received 25,995 measures of dfBeta), using the 'dfbetas' function from R \cite{rcoreteam2018r}. The measure of statistical leverage, then, of a given checklist was the sum of the absolute value of the dfBeta measures for each species (i.e., the sum of all 235 dfBetas). This measure of statistical leverage was thus a measure of a checklist's influence in understanding cumulative species' trends throughout the Greater Sydney Region.

\subsection*{Parameter calculation} After our model was fit from 2010---2018, we calculated the predicted parameters of interest for each day in 2018 (N=365). For each individual grid, at each of the grid sizes, we dynamically calculated the following parameters: (1) whether a grid cell had ever been sampled, (2) the distance to the nearest sampled grid cell, (3) the median sampling interval of a grid cell, (4) the median sampling interval of the nearest sampled grid cell, (5) days since the last sample in a grid cell, and (6) the duration of sampling in a grid cell --- most recent sample minus the earliest sampled date, and (7) the number of unique sampling days within the grid cell.

We then subsetted the leverage calculations (see above) for each of the days in 2018, given we know where people sampled, relative to the parameters for each of the grids on that day. We ran a linear regression, for each of the different grid sizes considered in the analysis, to investigate what parameters were of significant interest. Prior to modelling, duration was highly correlated with median sampling interval for the majority of the grid size analyses, and as such, was excluded from consdieration. Given the paramters' correlation varies among grid cell sizes, we need to ensure a robust, and simple model. All variables were standardized prior to modelling, ensuring that the effect sizes of the given parameters were meaningful. The response variable, dfBeta (i.e., value) was log-transformed prior to modelling to meet model assumptions. Thus, the final model included a log-tranformed dfBeta response variable, regressed against standardized median sampling interval, number of days sampled, days since the last sample, distance to the nearest sampled neighbor, and the neighbor's median sampling interval.

After our model was fit, we used the 'augment' function from the broom package \cite{robinson2018broom} to predict the expected leverage for every grid cell in the Greater Sydney Region, for every day. For grid cells which were unsampled, we assigned them the mean of the sampled grid cells, based on our lack of evidence that unsampled cells were significantly more valuable than sampled cells. This prediction was done for every day of 2018.

\subsection*{Data availability} All eBird data are freely available for download (https://ebird.org/data/download), but the necessary portion of the eBird basic dataset, along with spatial data, and code to reproduce our analyses are available at: GitHub repository.
}

\showmatmethods{} % Display the Materials and Methods section

\acknow{We thank the countless citizen scientists who are contributing data that is continuously increasing our collective knowledge of biodiversity.}

\showacknow{} % Display the acknowledgments section

% Bibliography
\bibliography{refs}

\end{document}
