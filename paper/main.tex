\documentclass[9pt,twocolumn,twoside,lineno]{pnas-new}
% Use the lineno option to display guide line numbers if required.

\templatetype{pnasresearcharticle} % Choose template 
% {pnasresearcharticle} = Template for a two-column research article
% {pnasmathematics} %= Template for a one-column mathematics article
% {pnasinvited} %= Template for a PNAS invited submission

\title{Optimizing biodiversity sampling by citizen scientists}

% Use letters for affiliations, numbers to show equal authorship (if applicable) and to indicate the corresponding author
\author[a,b,1]{Corey T. Callaghan}
\author[c,]{Alistair G. B. Poore} 
\author[b,a]{Richard E. Major}
\author[b,a]{Jodi J. L. Rowley}
\author[a,c]{William K. Cornwell}

\affil[a]{Centre for Ecosystem Science, School of Biological, Earth and Environmental Sciences, UNSW Sydney, Sydney, 2052, NSW, Australia}
\affil[b]{Australian Museum Research Institute, Australian Museum, Sydney, 2000, NSW, Australia}
\affil[c]{Ecology and Evolution Research Centre, School of Biological, Earth and Environmental Sciences, UNSW Sydney, Sydney, 2052, NSW, Australia}

% Please give the surname of the lead author for the running footer
\leadauthor{Callaghan} 

% Please add here a significance statement to explain the relevance of your work
\significancestatement{Citizen scientists are increasingly monitoring biodiversity, but we need to better understand how to optimally sample biodiversity in time and space for intended outcomes, relying on citizen science data.}

% Please include corresponding author, author contribution and author declaration information
\authorcontributions{Please provide details of author contributions here.}
\authordeclaration{The authors have no conflict of interest to declare.}

\correspondingauthor{\textsuperscript{1}To whom correspondence should be addressed. E-mail: c.callaghan@unsw.edu.au}

% Keywords are not mandatory, but authors are strongly encouraged to provide them. If provided, please include two to five keywords, separated by the pipe symbol, e.g:
\keywords{biodiversity $|$ conservation $|$ citizen science $|$ optimal sampling $|$ eBird} 

\begin{abstract}
Citizen science is important. yata yata yata
\end{abstract}

\dates{This manuscript was compiled on \today}
\doi{\url{www.pnas.org/cgi/doi/10.1073/pnas.XXXXXXXXXX}}

\begin{document}

\maketitle
\thispagestyle{firststyle}
\ifthenelse{\boolean{shortarticle}}{\ifthenelse{\boolean{singlecolumn}}{\abscontentformatted}{\abscontent}}{}

% If your first paragraph (i.e. with the \dropcap) contains a list environment (quote, quotation, theorem, definition, enumerate, itemize...), the line after the list may have some extra indentation. If this is the case, add \parshape=0 to the end of the list environment.
\dropcap{T}

This work is really important \cite{theobald2015global}.

\section*{Results}
\section*{Discussion}



\matmethods{Please describe your materials and methods here. This can be more than one paragraph, and may contain subsections and equations as required. Authors should include a statement in the methods section describing how readers will be able to access the data in the paper. 

\subsection*{eBird data}
We used eBird data....


\subsection*{Parameter calculation}
For 2018, we
}

\showmatmethods{} % Display the Materials and Methods section

\acknow{We thank the countless citizen scientists who are contributing data that is continuously increasing our collective knowledge of biodiversity.}

\showacknow{} % Display the acknowledgments section

% Bibliography
\bibliography{refs}

\end{document}
